\bf Lösung \rm

\begin{itemize}
\item[a)] Der Kanal hat $t=2$ Sendeantennen und $r=3$ Empfangsantennen.
\item[b)] Zur Bestimmung der Eigenwerte der Matrix
\begin{align*}
\matH^*\matH=\left(\begin{array}{ccc}
1 & 1 & -i \\
-i & i & 1  \\
\end{array}\right)\left(\begin{array}{cc}
1 & i \\
1 & -i \\
i & 1 \\
\end{array}\right)=\left(\begin{array}{cc}
3 & -i \\
i & 3 \\
\end{array}\right)
\end{align*}
berechnen wir das charakteristische Polynom
\begin{align*}
\left|\matH^*\matH-\lambda\matI_2\right|&=\left|\begin{array}{cc}
3-\lambda & -i \\
i & 3-\lambda \\
\end{array}\right| \\
&=(3-\lambda)(3-\lambda)-1 \\
&=\lambda^2-6\lambda +8 \\
&=(\lambda-4)(\lambda-2),
\end{align*}
daraus ergeben sich die Eigenwerte $\lambda_1=4$ und $\lambda_2=2$.
Mittels Waterfilling suchen wir nun
$\nu$ mit
\begin{align*}
\sum_{i=1}^2\left(\nu-\frac{\sigma^2}{\lambda_i}\right)^+=\left(\nu-\frac{4}{4}\right)^++\left(\nu-\frac{4}{4}\right)^+=L=11.
\end{align*}
Durch Weglassen des $^+$-Operators ergibt sich
\begin{align*}
(\nu-1)+(\nu-2)=2\nu-3=11,
\end{align*}
also $\nu=7$. Da für diesen Wert beide Summanden positiv sind, ist dies auch schon die richtige
Lösung, und die Kanalkapazität ist folglich
\begin{align*}
C=\sum_{i=1}^2\left(\ln\left(\frac{\nu\lambda_i}{\sigma^2}\right)\right)^+
=\ln\left(\frac{7\cdot4}{4}\right)^+ + \ln\left(\frac{7\cdot2}{4}\right)^+ 
=\ln\left(7\right) + \ln\left(\frac{7}{2}\right)\approx 3.199.
\end{align*}
\item[c)] Um die kapazitätserreichende Eingabeverteilung zu ermitteln, werden die Eigenvektoren
der Matrix $\matH^*\matH$ benötigt. Eigenvektoren zum Eigenwert $\lambda_1=4$ sind alle
Vektoren $\vecv$, die das Gleichungssystem
\begin{align*}
\left(\matH^*\matH-\lambda_1\cdot\matI_2\right)\vecv=\left(\begin{array}{cc}
-1 & -i \\
i & -1 \\
\end{array}\right)\left(\begin{array}{c}
v_1 \\ v_2 \\
\end{array}\right)=\veczero
\end{align*}
erfüllen. Man kann beispielsweise $v_1=1$ wählen und erhält durch Einsetzen in eine der
beiden Gleichungen den Wert $v_2=i$. Analog sind alle Eigenvektoren zum Eigenwert
$\lambda_2=2$ durch das Gleichungssystem
\begin{align*}
\left(\matH^*\matH-\lambda_2\cdot\matI_2\right)\vecw=\left(\begin{array}{cc}
1 & -i \\
i & 1 \\
\end{array}\right)\left(\begin{array}{c}
w_1 \\ w_2 \\
\end{array}\right)=\veczero
\end{align*}
bestimmt. Aus der Wahl $w_1=1$ folgt hier $w_2=-i$.
Normiert ergeben sich die Vektoren
\begin{align*}
\vecu_1=\frac{\vecv}{\|\vecv\|}=\frac{1}{\sqrt{2}}\left(\begin{array}{c}
1 \\ i \\
\end{array}\right) \qquad \text{und} \qquad \vecu_2=\frac{\vecw}{\|\vecw\|}=\frac{1}{\sqrt{2}}\left(\begin{array}{c}
1 \\ -i \\
\end{array}\right),
\end{align*}
die die Spalten der Matrix $\matU$ bilden.
Die gesuchte Kovarianzmatrix der Eingabeverteilung $\rvecX\sim\scndist(\veczero,\matQ)$ ist also
\begin{align*}
\matQ&=\matU\diag\left(\left(\nu-\frac{\sigma^2}{\lambda_i}\right)^+\right)\matU^* \\
&=\frac{1}{2}\left(\begin{array}{cc}
1 & 1 \\
i & -i \\
\end{array}\right)\left(\begin{array}{cc}
6 & 0 \\
0 & 5 \\
\end{array}\right)\left(\begin{array}{cc}
1 & -i \\
1 & i \\
\end{array}\right) \displaybreak[0] \\
&=\frac{1}{2}\left(\begin{array}{cc}
6 & 5 \\
6i & -5i \\
\end{array}\right)\left(\begin{array}{cc}
1 & -i \\
1 & i \\
\end{array}\right) \\
&=\frac{1}{2}\left(\begin{array}{cc} 
11 & -i \\
i & 11 \\
\end{array}\right).
\end{align*}
\item[d)] Der Parameter $\nu$ muss nun die Bedingung
\begin{align*}
\sum_{i=1}^2\left(\nu-\frac{\sigma^2}{\lambda_i}\right)^+=\left(\nu-\frac{4}{4}\right)^++\left(\nu-\frac{4}{2}\right)^+=L=\frac{1}{3}
\end{align*}
erfüllen. Der erste Ansatz ohne $^+$-Operator lautet
\begin{align*}
(\nu-1)+(\nu-2)=2\nu-3=\frac{1}{3}
\end{align*}
und ergibt $\nu=\frac{5}{3}$. Für diesen Wert ist allerdings $\nu-2<0$, es handelt sich also nicht um
eine gültige Lösung. Weglassen des zweiten Summanden liefert den Ansatz
\begin{align*}
(\nu-1)=\frac{1}{3},
\end{align*}
daraus folgt die gültige Lösung $\nu=\frac{4}{3}$. Die Kanalkapazität ist nun also
\begin{align*}
C=\sum_{i=1}^2\left(\ln\left(\frac{\nu\lambda_i}{\sigma^2}\right)\right)^+
=\ln(\frac{4}{3})+\underbrace{\left(\ln\left(\frac{4}{3}\right)\right)^+}_{=0}
=\ln(4) - \ln(3)
\approx 0.288.
\end{align*}
Sie wird erreicht für $\rvecX\sim\scndist(\veczero,\matQ)$ mit Kovarianzmatrix
\begin{align*}
\matQ&=\matU\diag\left(\left(\nu-\frac{7}{\lambda_i}\right)^+\right)\matU^* \\
&=\frac{1}{2}\left(\begin{array}{cc}
1 & 1 \\
i & -i \\
\end{array}\right)\left(\begin{array}{cc}
\frac{1}{3} & 0 \\
0 & 0 \\
\end{array}\right)\left(\begin{array}{cc}
1 & -i \\
1 & i \\
\end{array}\right) \displaybreak[0] \\
&=\frac{1}{2} \cdot \frac{1}{3}\left(\begin{array}{cc}
1 & -i \\
i & 1 \\
\end{array}\right)
=\left(\begin{array}{cc}
\frac{1}{6} & \frac{-i}{6} \\
\frac{i}{6} & \frac{1}{6} \\
\end{array}\right).
\end{align*}

\item[e)] $\left(\nu-\frac{\sigma^2}{\lambda_1}\right)^+ = \frac{L}{2}$ and $\left(\nu-\frac{\sigma^2}{\lambda_2}\right)^+ = \frac{L}{2}$.

\item[f)] $\left(\nu-\frac{\sigma^2}{\lambda_1}\right)^+ = L$ and $\left(\nu-\frac{\sigma^2}{\lambda_2}\right)^+ = 0$.

\end{itemize}

\newpage

%%%
%%%
%%%