% \iffalse meta-comment
%
% Copyright (C) 2015 by Niklas Koep <koep@ti.rwth-aachen.de>
%
% This file may be distributed and/or modified under the
% conditions of the LaTeX Project Public License, either version 1.2
% of this license or (at your option) any later version.
% The latest version of this license is in:
%
% http://www.latex-project.org/lppl.txt
%
% and version 1.2 or later is part of all distributions of LaTeX
% version 1999/12/01 or later.
%
% \fi
%
% \iffalse
%<package>\NeedsTeXFormat{LaTeX2e}
%<package>\ProvidesPackage{tiproblem}
%<package>[2015/09/13 v0.1 Macros to typeset problems and solutions]
%
%<*driver>
\documentclass{ltxdoc}
\GetFileInfo{tiproblem.dtx}
\def\@author{Niklas Koep \\ \texttt{koep@ti.rwth-aachen.de}}
\usepackage{tiproblem}
\EnableCrossrefs
\CodelineIndex
\RecordChanges
\begin{document}
  \title{The \textsf{tiproblem} package\thanks{This document corresponds to
    \textsf{tiproblem}~\fileversion, last revised on \filedate.}}
  \author{\@author}
  \maketitle
  \DocInput{tiproblem.dtx}
\end{document}
%</driver>
% \fi
%
% \CheckSum{0}
%
% \DoNotIndex{\newcommand,\newenvironment}
%
% This package implements various macros and environments to typeset problems
% and sample solutions for tutorials and exams. Additionally, it provides a set
% of macros that can be used to compile problem collection documents where
% sample solutions immediately follow problem descriptions.
%
% \section{Package Options}
% \DescribeMacro{english}
% This option makes sure any labels produced by macros of this package are
% properly translated to English.
%
% \section{Environments}
% \DescribeEnv{problem}
% The |problem| environment is used to typeset problem descriptions. The
% problem text inside this environment is prefixed by a generic label and the
% current value of the internal problem counter (cf. |\setproblemnumber|).
%
% \DescribeEnv{solution}
% The |solution| environment is the counterpart to the |problem| environment
% for typesetting solutions. Note that this environment uses its own internal
% counter which can be adjusted by hand if necessary (cf.
% |\setsolutionnumber|).
%
% \section{Macros}
% \DescribeMacro{\setproblemnumber}
% The |\setproblemnumber|\marg{number} macro can be used to modify the internal
% problem counter. By default, the counter starts at $1$. This macro is
% intended to be used if the problem count is suppposed to continuously
% increase across problem sheets as the semester progresses rather than being
% reset to $1$ for every problem sheet.
%
% \DescribeMacro{\setsolutionnumber}
% This macro behaves like |\setproblemnumber|, but modifies the internal
% solution counter instead of the problem counter.
%
% Note that the previous two macros only take effect if they are invoked in the
% document preamble as their values are expanded inside the |\AtBeginDocument|
% hook.
%
% \DescribeMacro{\setbasepath}
% For the TI lecture, all available problems and sample solutions are collected
% in a master document to ease the selection of problems for individual
% tutorial sessions. Each problem is contained in a dedicated file that resides
% in a folder whose name is of the form |sectionXX|, where |XX| denotes some
% section number. In order to include problems and sample solutions straight
% from the problem collection when typesetting problem sheets, the |tiproblem|
% package needs to know where to look for problem files. This is achieved with
% the |\setbasepath|\marg{path} macro which is used to point the package to the
% root directory of a problem collection tree.
%
% \DescribeMacro{\setsearchpath}
% When a problem and its solution are not part of a problem collection, it is
% possible to directly include \LaTeX\ files containing problem descriptions or
% sample solutions by using the |\insertproblemfromfile| or
% |\insertsolutionfromfile| macro, respectively (see below). By default, these
% files are assumed to reside in the same directory as the current document.
% The |\setsearchpath|\marg{path} macro can be used to modify this search path
% to allow including files from arbitrary directories.
%
% \DescribeMacro{\setsolutionsuffix}
% Assume that the \LaTeX\ source code for a problem is contained in the file
% |section13/012.tex| inside the master document directory tree. When
% typesetting the master document, it is assumed that the sample solution for
% this problem is located at |section13/012|\marg{suffix}|.tex|, where
% \marg{suffix} denotes the suffix of solution documents. By default, this
% suffix is |-solution| (note the leading dash). If a problem collection uses a
% different naming convention, the |\setsolutionsuffix|\marg{suffix} macro can
% be used to adjust the suffix accordingly.
%
% \DescribeMacro{\insertproblem}
% The |\insertproblem|\marg{section}\marg{problem} macro inserts problem
% \meta{problem}|.tex| from section \meta{section} of a problem collection into
% the document. The macro fails gracefully if presented with an invalid
% \meta{section}-\meta{problem} pair and outputs an error message in place of
% the problem text.
%
% \DescribeMacro{\insertproblemfromfile}
% By invoking |\insertproblemfromfile|\marg{filename}, the contents of the file
% \meta{filename}|.tex| can be directly included in the document without being
% part of a problem collection (cf. |\setsearchpath|).
%
% \DescribeMacro{\insertsolution}
% The |\insertsolution|\marg{section}\marg{problem} macro inserts the solution
% to problem \meta{problem} of section \meta{section} of a problem collection.
% Similar to the behavior of |\insertproblem|, this macro also fails
% gracefully.
%
% \DescribeMacro{\insertsolutionfromfile}
% The |\insertsolutionfromfile|\marg{filename} macro is the natural counterpart
% of the |\insertproblemfromfile| macro. It loads a sample solution from
% \meta{filename}\meta{-suffix}|.tex| found in the directory configured through
% |\setbasepath|.
%
% \StopEventually{\PrintIndex}
%
% \section{Implementation}
%    \begin{macrocode}
\usepackage{tienv}
\usepackage{timath}

\def\tiproblem@problemlabel{Aufgabe}
\def\tiproblem@solutionlabel{Lösung zu Aufgabe}
\DeclareOption{english}{
  \def\tiproblem@problemlabel{Problem}
  \def\tiproblem@solutionlabel{Solution of Problem}
}
\ProcessOptions\relax

\newcounter{tiproblem@ProblemCounter}[section]
\newcounter{tiproblem@SolutionCounter}[section]

\newenvironment{problem}{%
  \stepcounter{tiproblem@ProblemCounter}%
  \par%
  \textsf{%
    \pagebreak[3]%
    \textbf{\tiproblem@problemlabel\ \thetiproblem@ProblemCounter.}%
  }%
}{\vspace*{1cm}\par}

\newenvironment{solution}{%
  \stepcounter{tiproblem@SolutionCounter}%
  \par%
  \textsf{%
    \pagebreak[3]%
    \textbf{%
      \large{\tiproblem@solutionlabel\ \thetiproblem@SolutionCounter}%
    }%
  } \\[2mm]
}{\vspace*{1cm}\par}

%    \end{macrocode}
% Our counters are zero-based and incremented before they're evaluated. Hence
% the |#1-1| in the commands below. Note that arithmetic expressions like
% |#1-1| are made possible by the |calc| package included above.
%    \begin{macrocode}
\newcommand*{\setproblemnumber}[1]{%
  \setcounter{tiproblem@ProblemCounter}{#1-1}%
}
\newcommand*{\setsolutionnumber}[1]{%
  \setcounter{tiproblem@SolutionCounter}{#1-1}%
}

\newcommand{\setbasepath}[1]{\def\tiproblem@basepath{#1}}
\setbasepath{.}

\newcommand{\setsearchpath}[1]{\def\tiproblem@searchpath{#1}}
\setsearchpath{.}

\newcommand{\setsolutionsuffix}[1]{\def\tiproblem@solutionsuffix{#1}}
\setsolutionsuffix{-solution}

\newcommand{\insertproblem}[2]{
  \def\tiproblem@exercisepath{\tiproblem@basepath/section#1/#2.tex}
  \begin{problem}
    \IfFileExists{\tiproblem@exercisepath}{
      \subimport{\tiproblem@basepath/section#1/}{#2.tex}
    }{
      \begin{center}
        \textbf{-- \tiproblem@exercisepath\ not found --}
      \end{center}
    }
  \end{problem}
}

\newcommand{\insertproblemfromfile}[1]{
  \begin{problem}
    \subimport{\tiproblem@searchpath/}{#1.tex}
  \end{problem}
}

\newcommand{\insertsolution}[2]{
  \def\tiproblem@solutionpath{%
    \tiproblem@basepath/section#1/#2\tiproblem@solutionsuffix.tex%
  }
  \begin{solution}
    \IfFileExists{\tiproblem@solutionpath}{
      \subimport{\tiproblem@basepath/section#1/}{%
        #2\tiproblem@solutionsuffix.tex%
      }
    }{
      \begin{center}
        \textbf{-- \tiproblem@solutionpath\ not found --}
      \end{center}
    }
  \end{solution}
}

\newcommand{\insertsolutionfromfile}[1]{
  \begin{solution}
    \subimport{\tiproblem@searchpath/}{#1\tiproblem@solutionsuffix.tex}
  \end{solution}
}
%    \end{macrocode}
%
% \Finale
\endinput

