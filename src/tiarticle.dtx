% \iffalse meta-comment
%
% Copyright (C) 2015 by Niklas Koep <koep@ti.rwth-aachen.de>
%
% This file may be distributed and/or modified under the
% conditions of the LaTeX Project Public License, either version 1.2
% of this license or (at your option) any later version.
% The latest version of this license is in:
%
% http://www.latex-project.org/lppl.txt
%
% and version 1.2 or later is part of all distributions of LaTeX
% version 1999/12/01 or later.
%
% \fi
%
% \iffalse
%<*driver>
\ProvidesFile{tiarticle.dtx}
%</driver>
%<class>\NeedsTeXFormat{LaTeX2e}
%<class>\ProvidesClass{tiarticle}
%<*class>
  [2015/09/13 v0.1 Base class for all tilatex classes]
%</class>
%<*driver>
\documentclass{ltxdoc}
\EnableCrossrefs
\GetFileInfo{tiarticle.dtx}
\def\@author{Niklas Koep \\ \texttt{koep@ti.rwth-aachen.de}}
\CodelineIndex
\RecordChanges
\begin{document}
  \title{The \textsf{tiarticle} class\thanks{This document corresponds to
    \textsf{tiarticle}~\fileversion, dated \filedate.}}
  \author{\@author}
  \maketitle
  \DocInput{tiarticle.dtx}
\end{document}
%</driver>
% \fi
%
% \CheckSum{0}
%
% \GetFileInfo{tiarticle.dtx}
%
% \DoNotIndex{\newcommand,\newenvironment}
%
% The |tiarticle| class serves as base class for all document classes in
% |tilatex|. It provides common functionality like setting the language, the
% document title, typesetting the institute header, and so on.
%
% \section{Class Options}
% Since the |tiarticle| class is based on the standard |article| class, any
% option valid for |article| is also accepted by the |tiarticle| class. By
% default, we load the |article| class with the |a4paper| option.
%
% \DescribeMacro{english}
% This is a convenience option to set the language of the document. Passing
% this option will automatically call |babel|'s |\selectlanguage| macro at the
% beginning of the document. By default, we load the |babel| package with the
% |ngerman| option.
%
% \section{Macros}
% \DescribeMacro{\setdocumenttitle}
% Use |\setdocumenttitle|\marg{title} to change the document title to
% \meta{title}. The document title will automatically be printed in the
% document header. For instance, when typesetting a problem sheet for a
% specific lecture, use |\setdocumenttitle| to set the lecture name.

% \DescribeMacro{\getdocumenttitle}
% Use |\getdocumenttitle| to get the current document title. This is mainly
% needed for the |tiexam| class which needs to obtain the document title when
% creating the title page.
%
% \DescribeMacro{\setauthors}
% Use |\setauthors|\marg{authors} to specify the author list of the document.
% This list will be placed right below the header. It is intended to be used
% for problem sheets to list the lecture supervisor(s), as well the lecturer
% when typesetting exam documents.
%
% \StopEventually{\PrintIndex}
%
% \section{Implementation}
%
%    \begin{macrocode}
\RequirePackage{etoolbox}

\def\tiarticle@language{ngerman}
\def\tiarticle@logo{rwth_ti_de_rgb}
\DeclareOption{english}{
  \expandafter\def\expandafter\tiarticle@language\expandafter{\CurrentOption}
  \def\tiarticle@logo{rwth_ti_en_rgb}
}

\newcommand*{\tiarticle@authorstransform}[1]{\footnotesize #1}
\DeclareOption{@largeauthorlist}{
  \renewcommand*{\tiarticle@authorstransform}[1]{#1}
}

\DeclareOption*{\PassOptionsToClass{\CurrentOption}{article}}

\ProcessOptions\relax

\LoadClass[a4paper]{article}

\usepackage[colorlinks=true]{hyperref}
\usepackage{tiinternal}
\usepackage{tienv}

\pagestyle{empty}
\usepackage[
  left=2.2cm,
  right=2.2cm,
  top=2cm,
  bottom=2.5cm
]{geometry}
\parindent=0pt
\parskip=5pt

\newcommand*{\setdocumenttitle}[1]{
  \def\tiarticle@documenttitle{#1}
}
\setdocumenttitle{}

\newcommand*{\getdocumenttitle}{\tiarticle@documenttitle}

\newcommand*{\setauthors}[1]{
  \def\tiarticle@authors{#1}
}
\setauthors{}
%    \end{macrocode}
% The corporate identity logo has an $8$ mm border around its bounding box. We
% (arbitrarily) scale the logo by $0.625$ so it's approximately as large as it
% would be with |width=0.5\columnwidth|. This means we need to pad the document
% title and author list by $0.625 \cdot 8$ mm $= 5$ mm so the gaps match.
% This is what we need |\tiarticle@padding| for.
%    \begin{macrocode}
\def\tiarticle@padding{5mm}
\newenvironment{tightcenter}{%
  \setlength\topsep{0pt}
  \setlength\parskip{0pt}
  \begin{center}
}{\end{center}}
\newcommand{\tiarticle@inserttiheader}{
  \begin{minipage}{\columnwidth}
    \vspace{-8mm}
    \begin{minipage}[b]{0.49\columnwidth}
      \fontsize{12pt}{1em}\selectfont
      \textbf{\textsf{\tiarticle@documenttitle}}
      \vspace{\tiarticle@padding}
    \end{minipage}
    \hfill
    \includegraphics[scale=0.625]{\tiarticle@logo.pdf}
  \end{minipage}
  \hrule
  \vspace{\tiarticle@padding}
  \begin{tightcenter}
    \textbf{\tiarticle@authorstransform{\tiarticle@authors}}
  \end{tightcenter}
  \tiinternal@contentvpadding
}

\AfterEndPreamble{
  \expandafter\selectlanguage\expandafter{\tiarticle@language}
  \tiarticle@inserttiheader
}
%    \end{macrocode}
%
% \Finale
\endinput
