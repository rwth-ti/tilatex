\documentclass{ltxdoc}

\usepackage{hyperref}
\usepackage{parskip}

\usepackage{tidatetime}
\usepackage{tienv}
\usepackage{timath}
\usepackage{tiproblem}

\title{tilatex}
\author{Niklas Koep \\ \texttt{koep@ti.rwth-aachen.de}}

\hypersetup{colorlinks=true}

\EnableCrossrefs
\CodelineIndex
\RecordChanges

\begin{document}

\selectlanguage{english}

\maketitle

\begin{abstract}
  This document describes a collection of \LaTeX\ packages and classes
  providing common functionality used at the Institute for Theoretical
  Information Technology, RWTH Aachen University, Germany.
\end{abstract}

\section*{Introduction}
Even though \texttt{tilatex} is designed in a modular fashion, its main use is
facilitated by the various document classes that ship with \texttt{tilatex}.
While individual packages often expose user-facing macros, these macros are
generally intended to be used in conjunction with or by specific document
classes. It is therefore not recommended to load packages by hand but rather
base your document on one of the various classes described below. For instance,
when typesetting a problem sheet for a tutorial, it is recommended to use the
\texttt{titutorial} class, which automatically loads the \texttt{tiproblem}
package, rather than loading the package manually. One reason for this is that
\texttt{titutorial} wraps some of the lower level functionality of
\texttt{tiproblem} in a more convenient interface for document authors. For
this reason, the classes provided by \texttt{tilatex} are documented first.

\section*{TODO}
\begin{itemize}
  \item clean up internal package dependencies and purge the
    \texttt{tiinternal} package
  \item use \texttt{etoolbox}'s boolean option handling
    (\url{http://tex.stackexchange.com/questions/5894/latex-conditional-expression})
    rather than \texttt{\textbackslash newif} primitives
  \item add \texttt{\textbackslash problemexists} and \texttt{\textbackslash
    solutionexists} macros to \texttt{tiproblem}
\end{itemize}

\pagebreak
\tableofcontents

% Classes
\DocInclude{tiarticle}
\DocInclude{tiexam}
\DocInclude{tiproblemcollection}
\DocInclude{titutorial}
\DocInclude{tithesisnotice}

% Packages
\DocInclude{tidatetime}
\DocInclude{tienv}
% \DocInclude{tiinternal}
% \DocInclude{timath}
\DocInclude{tiproblem}

% Examples
% \DocInclude{examples}

\end{document}

