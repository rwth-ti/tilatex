% \iffalse meta-comment
%
% Copyright (C) 2015 by Niklas Koep <koep@ti.rwth-aachen.de>
%
% This file may be distributed and/or modified under the
% conditions of the LaTeX Project Public License, either version 1.2
% of this license or (at your option) any later version.
% The latest version of this license is in:
%
% http://www.latex-project.org/lppl.txt
%
% and version 1.2 or later is part of all distributions of LaTeX
% version 1999/12/01 or later.
%
% \fi
%
% \iffalse
%<package>\NeedsTeXFormat{LaTeX2e}
%<package>\ProvidesPackage{timath}
%<package>[2015/09/13 v0.1 Collection of math macros used in the TI script]
%
%<*driver>
\documentclass{ltxdoc}
\GetFileInfo{timath.dtx}
\def\@author{Niklas Koep \\ \texttt{koep@ti.rwth-aachen.de}}
\usepackage{timath}
\EnableCrossrefs
\CodelineIndex
\RecordChanges
\begin{document}
  \title{The \textsf{timath} package\thanks{This document corresponds to
    \textsf{timath}~\fileversion, last revised on \filedate.}}
  \author{\@author}
  \maketitle
  \DocInput{timath.dtx}
\end{document}
%</driver>
% \fi
%
% \CheckSum{0}
%
% \DoNotIndex{\newcommand,\newenvironment}
%
% This package provides a variety of mathematical macros used through the TI
% script, as well as a few necessary additions to compile the problem
% collection and tutorial sheets.
%
% \StopEventually{\PrintIndex}
%
% \section{Implementation}
%    \begin{macrocode}
\usepackage{amscd,amsfonts,amsmath,amssymb}
\usepackage{xspace}

\newcommand{\timath@safemath}[2]{\newcommand{#1}{\ensuremath{#2}\xspace}}

% Sets
\timath@safemath{\setR}{\mathbb{R}}
\timath@safemath{\setZ}{\mathbb{Z}}
\timath@safemath{\setN}{\mathbb{N}}
\timath@safemath{\setQ}{\mathbb{Q}}
\timath@safemath{\setC}{\mathbb{C}}
\timath@safemath{\setI}{\mathbb{I}}

\timath@safemath{\calS}{\mathcal{S}}
\timath@safemath{\calT}{\mathcal{T}}
\timath@safemath{\calC}{\mathcal{C}}
\timath@safemath{\calD}{\mathcal{D}}
\timath@safemath{\calF}{\mathcal{F}}

\timath@safemath{\borelset}{\mathfrak{B}}
\timath@safemath{\potset}{\mathfrak{P}}
\timath@safemath{\eventset}{\mathfrak{A}}
\timath@safemath{\sampleset}{\Omega}

% Discrete signal sets
\timath@safemath{\sigsetX}{\mathcal{X}}
\timath@safemath{\sigsetY}{\mathcal{Y}}
\timath@safemath{\sigsetZ}{\mathcal{Z}}
\timath@safemath{\sigsetW}{\mathcal{W}}
\timath@safemath{\sigsetA}{\mathcal{A}}
\timath@safemath{\sigsetB}{\mathcal{B}}
\timath@safemath{\sigsetC}{\mathcal{C}}

% Probability distributions
\DeclareMathOperator{\unidist}{U}
\DeclareMathOperator{\bindist}{Bin}
\DeclareMathOperator{\geodist}{Geo}
\DeclareMathOperator{\poidist}{Poi}
\DeclareMathOperator{\nordist}{N}
\DeclareMathOperator{\recdist}{R}
\DeclareMathOperator{\expdist}{Exp}
\DeclareMathOperator{\erldist}{Erl}
\DeclareMathOperator{\raydist}{Ray}
\DeclareMathOperator{\nakdist}{Nak}
\DeclareMathOperator{\gamdist}{\Gamma}
\DeclareMathOperator{\ricdist}{Rice}
\DeclareMathOperator{\logdist}{LogN}
\DeclareMathOperator{\cndist}{CN}
\DeclareMathOperator{\scndist}{SCN}

% Operators and functions
\DeclareMathOperator{\expop}{E}
\DeclareMathOperator{\varop}{Var}
\DeclareMathOperator{\covop}{Cov}
\DeclareMathOperator{\corop}{Corr}
% TODO: Replace the one use of this macro in the TI script.
% \DeclareMathOperator{\idop}{id}
\DeclareMathOperator{\traop}{tr}

\DeclareMathOperator{\fourop}{\mathfrak{F}}

\DeclareMathOperator{\imag}{Im}
\DeclareMathOperator{\real}{Re}

\DeclareMathOperator{\sinc}{si}

\DeclareMathOperator{\info}{I}
\DeclareMathOperator{\transinfo}{I}
\DeclareMathOperator{\entropy}{H}
\DeclareMathOperator{\kullback}{D}

\DeclareMathOperator{\ld}{ld}

\DeclareMathOperator{\diag}{diag}

\timath@safemath{\de}{\mathrm{d}}
\timath@safemath{\snr}{\mathrm{SNR}}

\newcommand{\lefto}{\mathopen{}\left}
\newcommand{\iid}{i.i.d.\@\xspace}

% Boldface math letters
\timath@safemath{\bZero}{\mathbf{0}}
\timath@safemath{\bma}{\mathbf{a}}
\timath@safemath{\bmb}{\mathbf{b}}
\timath@safemath{\bmc}{\mathbf{c}}
\timath@safemath{\bmd}{\mathbf{d}}
\timath@safemath{\bme}{\mathbf{e}}
\timath@safemath{\bmf}{\mathbf{f}}
\timath@safemath{\bmg}{\mathbf{g}}
\timath@safemath{\bmh}{\mathbf{h}}
\timath@safemath{\bmi}{\mathbf{i}}
\timath@safemath{\bmj}{\mathbf{j}}
\timath@safemath{\bmk}{\mathbf{k}}
\timath@safemath{\bml}{\mathbf{l}}
\timath@safemath{\bmm}{\mathbf{m}}
\timath@safemath{\bmn}{\mathbf{n}}
\timath@safemath{\bmo}{\mathbf{o}}
\timath@safemath{\bmp}{\mathbf{p}}
\timath@safemath{\bmq}{\mathbf{q}}
\timath@safemath{\bmr}{\mathbf{r}}
\timath@safemath{\bms}{\mathbf{s}}
\timath@safemath{\bmt}{\mathbf{t}}
\timath@safemath{\bmu}{\mathbf{u}}
\timath@safemath{\bmv}{\mathbf{v}}
\timath@safemath{\bmw}{\mathbf{w}}
\timath@safemath{\bmx}{\mathbf{x}}
\timath@safemath{\bmy}{\mathbf{y}}
\timath@safemath{\bmz}{\mathbf{z}}
\timath@safemath{\bmnu}{\mathbf{\nu}}

% Symbol definitions as used in the TI script (also see
% http://latex-project.org/guides/fntguide.pdf):
%   - OML == font encoding (TeX math italic)
%   - cmm == font family (computer modern math italic)
%   - b == font series (bold)
%   - it == font shape (italic)

% Boldface italic math letters
\DeclareSymbolFont{BoldMath}{OML}{cmm}{b}{it}
\DeclareMathSymbol{\boa}{\mathalpha}{BoldMath}{'141}
\DeclareMathSymbol{\bob}{\mathalpha}{BoldMath}{'142}
\DeclareMathSymbol{\boc}{\mathalpha}{BoldMath}{'143}
\DeclareMathSymbol{\bod}{\mathalpha}{BoldMath}{'144}
\DeclareMathSymbol{\boe}{\mathalpha}{BoldMath}{'145}
\DeclareMathSymbol{\bog}{\mathalpha}{BoldMath}{'147}
\DeclareMathSymbol{\boh}{\mathalpha}{BoldMath}{'150}
\DeclareMathSymbol{\bom}{\mathalpha}{BoldMath}{'155}
\DeclareMathSymbol{\bon}{\mathalpha}{BoldMath}{'156}
\DeclareMathSymbol{\bop}{\mathalpha}{BoldMath}{'160}
\DeclareMathSymbol{\boq}{\mathalpha}{BoldMath}{'161}
\DeclareMathSymbol{\bor}{\mathalpha}{BoldMath}{'162}
\DeclareMathSymbol{\bos}{\mathalpha}{BoldMath}{'163}
\DeclareMathSymbol{\bot}{\mathalpha}{BoldMath}{'164}
\DeclareMathSymbol{\bou}{\mathalpha}{BoldMath}{'165}
\DeclareMathSymbol{\bov}{\mathalpha}{BoldMath}{'166}
\DeclareMathSymbol{\bow}{\mathalpha}{BoldMath}{'167}
\DeclareMathSymbol{\boxx}{\mathalpha}{BoldMath}{'170}
\DeclareMathSymbol{\boy}{\mathalpha}{BoldMath}{'171}
\DeclareMathSymbol{\boz}{\mathalpha}{BoldMath}{'172}
\DeclareMathSymbol{\bolambda}{\mathalpha}{BoldMath}{'025}
\DeclareMathSymbol{\bomu}{\mathalpha}{BoldMath}{'026}
\DeclareMathSymbol{\bonu}{\mathalpha}{BoldMath}{'027}

% Boldface italic capital math letters
\DeclareMathSymbol{\boA}{\mathalpha}{BoldMath}{'101}
\DeclareMathSymbol{\boB}{\mathalpha}{BoldMath}{'102}
\DeclareMathSymbol{\boC}{\mathalpha}{BoldMath}{'103}
\DeclareMathSymbol{\boD}{\mathalpha}{BoldMath}{'104}
\DeclareMathSymbol{\boE}{\mathalpha}{BoldMath}{'105}
\DeclareMathSymbol{\boF}{\mathalpha}{BoldMath}{'106}
\DeclareMathSymbol{\boG}{\mathalpha}{BoldMath}{'107}
\DeclareMathSymbol{\boH}{\mathalpha}{BoldMath}{'110}
\DeclareMathSymbol{\boI}{\mathalpha}{BoldMath}{'111}
\DeclareMathSymbol{\boK}{\mathalpha}{BoldMath}{'113}
\DeclareMathSymbol{\boM}{\mathalpha}{BoldMath}{'115}
\DeclareMathSymbol{\boO}{\mathalpha}{BoldMath}{'117}
\DeclareMathSymbol{\boP}{\mathalpha}{BoldMath}{'120}
\DeclareMathSymbol{\boQ}{\mathalpha}{BoldMath}{'121}
\DeclareMathSymbol{\boR}{\mathalpha}{BoldMath}{'122}
\DeclareMathSymbol{\boS}{\mathalpha}{BoldMath}{'123}
\DeclareMathSymbol{\boT}{\mathalpha}{BoldMath}{'124}
\DeclareMathSymbol{\boU}{\mathalpha}{BoldMath}{'125}
\DeclareMathSymbol{\boV}{\mathalpha}{BoldMath}{'126}
\DeclareMathSymbol{\boW}{\mathalpha}{BoldMath}{'127}
\DeclareMathSymbol{\boX}{\mathalpha}{BoldMath}{'130}
\DeclareMathSymbol{\boY}{\mathalpha}{BoldMath}{'131}
\DeclareMathSymbol{\boZ}{\mathalpha}{BoldMath}{'132}

% Boldface capital math letters
\timath@safemath{\bA}{\mathbf{A}}
\timath@safemath{\bB}{\mathbf{B}}
\timath@safemath{\bC}{\mathbf{C}}
\timath@safemath{\bD}{\mathbf{D}}
\timath@safemath{\bE}{\mathbf{E}}
\timath@safemath{\bF}{\mathbf{F}}
\timath@safemath{\bG}{\mathbf{G}}
\timath@safemath{\bH}{\mathbf{H}}
\timath@safemath{\bI}{\mathbf{I}}
\timath@safemath{\bJ}{\mathbf{J}}
\timath@safemath{\bK}{\mathbf{K}}
\timath@safemath{\bL}{\mathbf{L}}
\timath@safemath{\bM}{\mathbf{M}}
\timath@safemath{\bN}{\mathbf{N}}
\timath@safemath{\bO}{\mathbf{O}}
\timath@safemath{\bP}{\mathbf{P}}
\timath@safemath{\bQ}{\mathbf{Q}}
\timath@safemath{\bR}{\mathbf{R}}
\timath@safemath{\bS}{\mathbf{S}}
\timath@safemath{\bT}{\mathbf{T}}
\timath@safemath{\bU}{\mathbf{U}}
\timath@safemath{\bV}{\mathbf{V}}
\timath@safemath{\bW}{\mathbf{W}}
\timath@safemath{\bX}{\mathbf{X}}
\timath@safemath{\bY}{\mathbf{Y}}
\timath@safemath{\bZ}{\mathbf{Z}}
\timath@safemath{\bSigma}{\mathbf{\Sigma}}
\timath@safemath{\bLambda}{\mathbf{\Lambda}}
\timath@safemath{\bGamma}{\mathbf{\Gamma}}
\timath@safemath{\bOne}{\mathbf{1}}
\timath@safemath{\bPi}{\mathbf{\Pi}}

% Matrices
\timath@safemath{\matA}{\boA}
\timath@safemath{\matB}{\boB}
\timath@safemath{\matC}{\boC}
\timath@safemath{\matD}{\boD}
\timath@safemath{\matE}{\boE}
\timath@safemath{\matF}{\boF}
\timath@safemath{\matG}{\boG}
\timath@safemath{\matH}{\boH}
\timath@safemath{\matI}{\boI}
\timath@safemath{\matJ}{\boJ}
\timath@safemath{\matK}{\boK}
\timath@safemath{\matL}{\boL}
\timath@safemath{\matM}{\boM}
\timath@safemath{\matN}{\boN}
\timath@safemath{\matO}{\boO}
\timath@safemath{\matP}{\boP}
\timath@safemath{\matQ}{\boQ}
\timath@safemath{\matR}{\boR}
\timath@safemath{\matS}{\boS}
\timath@safemath{\matT}{\boT}
\timath@safemath{\matU}{\boU}
\timath@safemath{\matV}{\boV}
\timath@safemath{\matW}{\boW}
\timath@safemath{\matX}{\boX}
\timath@safemath{\matY}{\boY}
\timath@safemath{\matZ}{\boZ}
\timath@safemath{\matZero}{\bZero}
\timath@safemath{\matSigma}{\bSigma}
\timath@safemath{\matLambda}{\bLambda}
\timath@safemath{\matGamma}{\bGamma}
\timath@safemath{\matOne}{\bOne}
\timath@safemath{\matPi}{\bPi}

% Vectors
\timath@safemath{\veca}{\boa}
\timath@safemath{\vecb}{\bob}
\timath@safemath{\vecc}{\boc}
\timath@safemath{\vecd}{\bod}
\timath@safemath{\vece}{\boe}
\timath@safemath{\vecf}{\bof}
\timath@safemath{\vecg}{\bog}
\timath@safemath{\vech}{\boh}
\timath@safemath{\veci}{\boi}
\timath@safemath{\vecj}{\boj}
\timath@safemath{\veck}{\bok}
\timath@safemath{\vecl}{\bol}
\timath@safemath{\vecm}{\bom}
\timath@safemath{\vecn}{\bon}
\timath@safemath{\veco}{\boo}
\timath@safemath{\vecp}{\bop}
\timath@safemath{\vecq}{\boq}
\timath@safemath{\vecr}{\bor}
\timath@safemath{\vecs}{\bos}
\timath@safemath{\vect}{\bot}
\timath@safemath{\vecu}{\bou}
\timath@safemath{\vecv}{\bov}
\timath@safemath{\vecw}{\bow}
\timath@safemath{\vecx}{\boxx}
\timath@safemath{\vecy}{\boy}
\timath@safemath{\vecz}{\boz}
\timath@safemath{\veczero}{\bZero}
\timath@safemath{\veclambda}{\bolambda}
\timath@safemath{\vecmu}{\bomu}
\timath@safemath{\vecnu}{\bonu}

% Random vectors
\timath@safemath{\rvecA}{\boA}
\timath@safemath{\rvecB}{\boB}
\timath@safemath{\rvecC}{\boC}
\timath@safemath{\rvecD}{\boD}
\timath@safemath{\rvecE}{\boE}
\timath@safemath{\rvecF}{\boF}
\timath@safemath{\rvecG}{\boG}
\timath@safemath{\rvecH}{\boH}
\timath@safemath{\rvecI}{\boI}
\timath@safemath{\rvecJ}{\boJ}
\timath@safemath{\rvecK}{\boK}
\timath@safemath{\rvecL}{\boL}
\timath@safemath{\rvecM}{\boM}
\timath@safemath{\rvecN}{\boN}
\timath@safemath{\rvecO}{\boO}
\timath@safemath{\rvecP}{\boP}
\timath@safemath{\rvecQ}{\boQ}
\timath@safemath{\rvecR}{\boR}
\timath@safemath{\rvecS}{\boS}
\timath@safemath{\rvecT}{\boT}
\timath@safemath{\rvecU}{\boU}
\timath@safemath{\rvecV}{\boV}
\timath@safemath{\rvecW}{\boW}
\timath@safemath{\rvecX}{\boX}
\timath@safemath{\rvecY}{\boY}
\timath@safemath{\rvecZ}{\boZ}
\timath@safemath{\rvecZero}{\bZero}

% Probability vectors
\timath@safemath{\probp}{\bmp}
\timath@safemath{\probq}{\bmq}

%%%%%%%%%%%%%%%%%%%%%%%%%%%%%%%%%%%%%%%%%%%
% ADDITIONAL MACROS USED FOR TI TUTORIALS %
%%%%%%%%%%%%%%%%%%%%%%%%%%%%%%%%%%%%%%%%%%%

% TODO: Implement the macros below on top of the math macros from the TI
%       script.

% Vector/matrix operators
\newcommand*{\conj}{^\ast} % Conjugate
\newcommand*{\transp}{'} % Transpose
\newcommand*{\inv}{^{-1}} % Inverse

\newcommand*{\probop}{P} % Probability measure
\DeclareMathOperator{\pcovop}{PCov}
\DeclareMathOperator{\sgnop}{sgn}
\DeclareMathOperator{\entop}{H}
\DeclareMathOperator{\miop}{I}
\DeclareMathOperator{\kulop}{D} % Kullback-Leibler divergence
\DeclareMathOperator{\transop}{I} % Transinformation (mutual information)

\DeclareMathOperator{\rdist}{R}
\newcommand*{\Bin}[2]{\ensuremath{\operatorname{Bin}\left(#1,#2\right)}}
\newcommand*{\Corr}[2]{\ensuremath{\operatorname{Corr}\left(#1,#2\right)}}
\newcommand*{\Cov}[2]{\ensuremath{\operatorname{Cov}\left(#1,#2\right)}}
\newcommand*{\Erl}[2]{\ensuremath{\operatorname{Erl}\left(#1,#2\right)}}
\newcommand*{\Exp}[1]{\ensuremath{\operatorname{Exp}\left(#1\right)}}
\newcommand*{\Geo}[1]{\ensuremath{\operatorname{Geo}\left(#1\right)}}
\newcommand*{\Nak}[2]{\ensuremath{\operatorname{Nak}\left(#1,#2\right)}}
\newcommand*{\Nor}[2]{\ensuremath{\operatorname{N}\left(#1,#2\right)}}
\newcommand*{\Poi}[1]{\ensuremath{\operatorname{Poi}\left(#1\right)}}
\newcommand*{\Ray}[1]{\ensuremath{\operatorname{Ray}\left(#1\right)}}

% Sets/fields
\newcommand*{\C}{\ensuremath{\mathbb{C}}}
\newcommand*{\F}{\ensuremath{\mathbb{F}}}
\newcommand*{\N}{\ensuremath{\mathbb{N}}}
\newcommand*{\R}{\ensuremath{\mathbb{R}}}
\newcommand*{\RR}{\mathbb{R}}
\newcommand*{\Rn}{\ensuremath{\mathbb{R}_-}}
\newcommand*{\Rp}{\ensuremath{\mathbb{R}_+}}
\newcommand*{\Z}{\ensuremath{\mathbb{Z}}}
\newcommand*{\cC}{\ensuremath{\mathcal{C}}}
\newcommand*{\cK}{\ensuremath{\mathcal{K}}}
\newcommand*{\cM}{\ensuremath{\mathcal{M}}}
\newcommand*{\cS}{\ensuremath{\mathcal{S}}}
\newcommand*{\cT}{\ensuremath{\mathcal{T}}}
\newcommand*{\cX}{\ensuremath{\mathcal{X}}}
\newcommand*{\cY}{\ensuremath{\mathcal{Y}}}
\newcommand*{\GF}{\mathbb{G}F}

\DeclareMathOperator{\repart}{Re}
\DeclareMathOperator{\impart}{Im}

\newcommand{\abs}[1]{\ensuremath{\left|#1\right|}}

\newcommand*{\Are}{\repart(\matA)}
\newcommand*{\Aim}{\impart(\matA)}
\newcommand*{\Ainvre}{\repart(\matA^{-1})}
\newcommand*{\Ainvim}{\impart(\matA^{-1})}
\newcommand*{\Bre}{\repart(\matB)}
\newcommand*{\Bim}{\impart(\matB)}
\newcommand*{\Binvre}{\repart(\matB^{-1})}
\newcommand*{\Binvim}{\impart(\matB^{-1})}
\newcommand*{\ABre}{\repart(\matA\matB)}
\newcommand*{\ABim}{\impart(\matA\matB)}

\DeclareMathSymbol{\bozero}{\mathalpha}{BoldMath}{'060}

% XXX: This defines |\idop| as the indicator function, rather than the identity
%      map.
\DeclareMathOperator{\idop}{\mathbb{I}}

%%%%%%%%%%%%%%%%%%%%%%%%%%%%%%%%%%%%%%%%%%%%%%%%%%%%%
% ADDITIONAL MACROS USED FOR OPTIMIZATION TUTORIALS %
%%%%%%%%%%%%%%%%%%%%%%%%%%%%%%%%%%%%%%%%%%%%%%%%%%%%%

% Colors
\definecolor {GREY} {rgb} {0.7,0.7,0.7}
\definecolor {LIGHTGREY} {rgb} {0.8,0.8,0.8}
\newcommand{\bl}{\color{blue}}
\newcommand{\black}{\color{black}}
\newcommand{\blue}{\color{blue}}
\newcommand{\bk}{\color{black}}
\newcommand{\green}{\color{green}}
\newcommand{\grey}{\color{GREY}}
\newcommand{\red}{\color{red}}

% Calliographic letters / sets
\newcommand{\calA}{\mathcal{A}}
\newcommand{\calE}{\mathcal{E}}
\newcommand{\calH}{\mathcal{H}}
\newcommand{\calI}{\mathcal{I}}
\newcommand{\calK}{\mathcal{K}}
\newcommand{\calP}{\mathcal{P}}
\newcommand{\calQ}{\mathcal{Q}}
\newcommand{\calR}{\mathcal{R}}
\newcommand{\calV}{\mathcal{V}}
\newcommand{\calW}{\mathcal{W}}
\newcommand{\calX}{\mathcal{X}}

% Operators
\newcommand{\argmax}{\operatorname{argmax}}
\newcommand{\argmin}{\operatorname{argmin}}
\DeclareMathOperator{\argop}{\mathbb{I}}
\newcommand{\bd}{\ensuremath{\mathrm{bd}}}
\newcommand{\BER}{\ensuremath{\mathrm{BER}}}
\newcommand{\ch}{\ensuremath{\operatorname{char}}}
\newcommand{\CNR}{\ensuremath{\mathrm{CNR}}}
\DeclareMathOperator{\conv}{conv}
\newcommand{\dist}{\ensuremath{\mathrm{dist}}}
\newcommand{\dom}{\ensuremath{\mathrm{dom}}}
\newcommand{\E}{\operatorname{E}}
\DeclareMathOperator{\entr}{entr}
\DeclareMathOperator{\epi}{epi}
\newcommand{\Epi}{\operatorname{epi}}
\DeclareMathOperator{\hyp}{hyp}
\newcommand{\Ima}{\ensuremath{\mathrm{Im}}}
\newcommand{\inter}{\ensuremath{\mathrm{int}}}
\DeclareMathOperator{\interior}{int}
\newcommand{\Ln}{\operatorname{Ln}}
\newcommand{\Log}{\operatorname{Log}}
\newcommand{\No}{\operatorname{N}}
\newcommand{\NP}{\operatorname{NP}}
\newcommand{\PP}{\operatorname{PP}}
\DeclareMathOperator{\rank}{rank}
\newcommand{\Rea}{\ensuremath{\mathrm{Re}}}
\newcommand{\sign}{\operatorname{sign}}
\newcommand{\SIR}{\ensuremath{\mathrm{SIR}}}
\DeclareMathOperator{\sumMat}{sum}
\newcommand{\tra}{\operatorname{tr}}
\newcommand{\U}{\operatorname{U}}
\newcommand{\Var}{\operatorname{Var}}

% Functions
\newcommand{\Her}[1]{\ensuremath{#1^{\mathsf H}}}
\newcommand{\herm}[1]{#1^{\mathsf{H}}}
\newcommand{\jacobi}[2]{\ensuremath{\left(\frac{#1}{#2}\right)}}
\newcommand{\norm}[2]{\left\|#1\right\|_{#2}}
\newcommand{\tr}[1]{\ensuremath{#1^{\mathsf T}}}
\newcommand{\tran}[1]{#1^{T}}
\newcommand{\vectornorm}[1]{\left|\left|#1\right|\right|}

% Bars
\newcommand{\be}{\bar{\eta}}
\newcommand{\bg}{\bar{g}}
\newcommand{\bm}{\bar{\mu}}
\newcommand{\bx}{\bar{x}}

% Tildes
\newcommand{\te}{\tilde{\eta}}
\newcommand{\tg}{\tilde{g}}
\newcommand{\tm}{\tilde{\mu}}

% Greek Symbols
\newcommand{\al}{\alpha}

% Misc math symbols
\newcommand{\aequiv}{\ensuremath{\Leftrightarrow}}
\newcommand{\follows}{\ensuremath{\Rightarrow}}
\newcommand{\I}{\mathbb I}   % Indikatorfunktion
\newcommand{\ie}{i.e.}
\newcommand{\Nn}{\N_{0}}
\newcommand{\PPl}{\PP(\lambda)}
\newcommand{\Q}{\mathbb Q}
\newcommand{\ri}{1 \leq i \leq T}
\newcommand{\Ss}{\mathbb{S}}
\newcommand{\WRaum}{$(\Omega,\A,\PMenge)$}
\newcommand{\ZZ}{\mathbb Z}

% Bold letters
\newcommand{\bc}{\ensuremath{\mathbf{c}}}
\newcommand{\br}{\ensuremath{\mathbf{r}}}
\newcommand{\bfsd}{\mathbf{s'}}
\newcommand{\bfu}{\mathbf{u}}
\newcommand{\bfun}{\mathbf{u^0}}
\newcommand{\bof}{\mathbf{f}}
\newcommand{\boomega}{{\bf \omega}}
\newcommand{\W}{\mathbf{W}}

\DeclareMathSymbol{\boC}{\mathalpha}{BoldMath}{'103}

% Bold math functions
\newcommand{\boone}{\ensuremath{{\mathbf{1}}}}

% Bold Greek Symbols
\DeclareMathSymbol{\boalpha}{\mathalpha}{BoldMath}{'013}
\DeclareMathSymbol{\bobeta}{\mathalpha}{BoldMath}{'014}
\DeclareMathSymbol{\bodelta}{\mathalpha}{BoldMath}{'016}
\DeclareMathSymbol{\boepsilon}{\mathalpha}{BoldMath}{'042}
\DeclareMathSymbol{\bogamma}{\mathalpha}{BoldMath}{'015}
\DeclareMathSymbol{\bolambda}{\mathalpha}{BoldMath}{'025}
\DeclareMathSymbol{\bomu}{\mathalpha}{BoldMath}{'026}
\DeclareMathSymbol{\bopi}{\mathalpha}{BoldMath}{'031}
\DeclareMathSymbol{\bophi}{\mathalpha}{BoldMath}{'047}
\DeclareMathSymbol{\bosigma}{\mathalpha}{BoldMath}{'033}
\DeclareMathSymbol{\botau}{\mathalpha}{BoldMath}{'034}
\DeclareMathSymbol{\boxi}{\mathalpha}{BoldMath}{'030}
\DeclareMathSymbol{\boGamma}{\mathalpha}{BoldMath}{'000}
\DeclareMathSymbol{\boDelta}{\mathalpha}{BoldMath}{'001}
\DeclareMathSymbol{\boLambda}{\mathalpha}{BoldMath}{'003}
\DeclareMathSymbol{\boPi}{\mathalpha}{BoldMath}{'005}
\DeclareMathSymbol{\boSigma}{\mathalpha}{BoldMath}{'006}
\DeclareMathSymbol{\boPhi}{\mathalpha}{BoldMath}{'010}
\DeclareMathSymbol{\boPsi}{\mathalpha}{BoldMath}{'011}

% Matrices
\newcommand{\matnull}{\boldsymbol{0}}
\newcommand{\mateins}{\boldsymbol{1}}

% Miscellaneous
\newsavebox\MBox
\newcommand\Cline[2][red]{{\sbox\MBox{$#2$}%
  \rlap{\usebox\MBox}\color{#1}\rule[-1.2\dp\MBox]{\wd\MBox}{0.5pt}}}

% References
\newcommand{\figref}[1]{Figure~\ref{fig:#1}}

% Abbreviations
\newcommand{\Xopt}{\ensuremath{\calX_\text{opt}}}

% Tikz
\tikzset{
  schraffiert/.style={pattern=horizontal lines,pattern color=#1},
  schraffiert/.default=black
}
\tikzset{
  schraffiertV/.style={pattern=vertical lines,pattern color=#1},
  schraffiertV/.default=black
}
%    \end{macrocode}
%
% \Finale
\endinput

